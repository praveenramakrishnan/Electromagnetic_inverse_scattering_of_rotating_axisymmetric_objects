\documentclass{ieeeaccess}
\usepackage{cite}
\usepackage{amsmath,amssymb,amsfonts}
\usepackage{algorithmic}
\usepackage{graphicx}
\usepackage{epstopdf}
\graphicspath{{../figures/}}                                                       

\usepackage{textcomp}
\def\BibTeX{{\rm B\kern-.05em{\sc i\kern-.025em b}\kern-.08em
    T\kern-.1667em\lower.7ex\hbox{E}\kern-.125emX}}
\begin{document}
\history{Date of publication xxxx 00, 0000, date of current version xxxx 00, 0000.}
\doi{10.1109/ACCESS.2017.DOI}

\title{Electromagnetic inverse scattering of rotating axisymmetric objects}
\author{\uppercase{Praveen Kalarickel Ramakrishnan}, 
\uppercase{Mario Rene Clemente Vargas},
\uppercase{and Mirco Raffetto}}

\address{Department of Electrical, Electronic, Telecommunications 
Engineering and Naval Architecture, University of Genoa, Genoa, Italy}

\markboth
{Kalarickel Ramakrishnan \headeretal: Electromagnetic inverse scattering of rotating axisymmetric objects}
{Kalarickel Ramakrishnan \headeretal: Electromagnetic inverse scattering of rotating axisymmetric objects}

\corresp{Corresponding author: Praveen Kalarickel Ramakrishnan (e-mail: pravin.nitc@gmail.com).}

\begin{abstract}
The electromagnetic inverse scattering problem for rotating axisymmetric objects is studied.
A modification of a previously proposed two-step algorithm is adopted to obtain the first solutions to the
problems of interest. In the first step, the forward solver is employed assuming zero rotating speed and
the geometric and dielectric parameters are reconstructed by minimizing the cost function. In the second
step, the values from the first step are used to determine the rotating speed. Numerical results for this type
of inverse problems are provided for the first time by considering test cases with rotating homogeneous
sphere and torus. The two-step algorithm is compared against the general inversion algorithm that relies on
global optimization considering all unknown variables simultaneously. It is demonstrated that the proposed
algorithm outperforms the general inversion procedure for all speeds of practical interest. 
The results are analyzed for noisy data in the near-field and far-field.
\end{abstract}

\begin{keywords}
Electromagnetic scattering, inverse problems, moving media, stochastic optimization,
remote sensing, rotating axisymmetric objects.
\end{keywords}

\titlepgskip=-15pt

\maketitle

\section{Introduction} \label{sec:introduction}
The electromagnetic problems involving moving objects are
important in numerous applications like astrophysics, nuclear
physics, plasma physics and engineering \cite{dezutterrotatingsphere}, 
\cite{vanbladel}, \cite{yeh_opu_su_movingplasmahalfspace}, 
\cite{yan_1996_mass_flow_measurement}, \cite{li2020electromagnetic}.
Although the most general problems need to be formulated
in time domain \cite{vanbladel}, there are important applications where
the motion of the objects is such that a frequency domain
formulation is feasible \cite{cheng_kong_covariant}. 
The moving media behave as
bianisotropic materials in the laboratory frame even when they are
isotropic in the rest frame \cite{noiscatteringellissimovimento}.

A very important class of the problems involves rotating
axisymmetric objects which can be studied using three-dimensional 
time-harmonic models \cite{dezutterrotatingsphere}, 
\cite{kalarickel2020well}. A set of sufficient 
conditions for well posedness and the finite element approximability
of the forward problem of calculating the fields
were established in \cite{kalarickel2020well}.
In some of the indicated problems, the
rotating velocity is not known and it is of particular interest
to be able to detect it using inverse scattering procedures.
However, to the best of the authors’ knowledge, there are no
results on the detection of the rotating velocity from scattered
field data.

In \cite{noiricostruzioneepsrebeta}  the authors developed 
a two-step algorithm for the
reconstruction of the velocity of two-dimensional problems
involving axially moving cylinders when the speeds are not
too large. The idea was based on the fact that for a particular
polarization of incident field (TM or TE), one of the polarizations
(cross-polarized component) is absent in the scattered
field when there is no motion and varies linearly with the
velocity for small speeds. In addition, the other polarization
of the scattered field (co-polarized component) is affected
only slightly (second order effect) due to motion. This led
to the algorithm in which the co-polarized component of
the scattered field is used in the first step to reconstruct the
dielectric and geometric properties of the media by ignoring
the effects of motion. The values deduced with the first step were then
used in the second step to find the axial speed from the cross-
polarized component. The authors were able to demonstrate
the efficiency of the two-step procedure over directly using
the whole field in a single step to simultaneously reconstruct
all the unknowns. 
The better performance of the two-step
procedure stems from the fact that the complexity of the
inversion algorithm increases much faster than linear with
the number of unknowns and hence the solution using two
simpler steps is much more efficient than solving the full
problem in a single step.
Further, in \cite{noinoiselimitation}, 
the authors studied the limitations
of the reconstruction algorithm when the measured
data are noisy and the sensors have limited capabilities.

In general three-dimensional problems, the effects of bianisotropy
are more complex \cite{kalarickel2020well}, 
\cite{kalarickel2020three} and it could be difficult
or even impossible to isolate field components able to provide
so clear indication on the velocity field. 
However, the first numerical
results for the fields in the presence of motion were
presented in \cite{kalarickel2020well} and they indicate that the effect due to 
motion is not too large even for relatively high values of rotating speeds.
Moreover, even for higher velocity values, the reconstruction of the 
geometric and dielectric unknowns by ignoring the motion can be even better 
than using the general algorithm since having a reduced number of unknowns 
not only speeds up the solution process but also gives improved precision.
Therefore, in this case, we propose to define a cost function using all the
components of the fields in the first step which is used to
reconstruct the geometric and dielectric properties assuming
that the media is under rest. In the second step, the values from
the first step are used to evaluate the speed of rotation. For
small rotating speed, the numerical results demonstrate that
such an approach is effective and fast compared to inverting
all the unknowns in a single step. The numerical results are
obtained for a test problem involving a homogeneous rotating
sphere which admits an analytic solution \cite{dezutterrotatingsphere}. 
The robustness of the results is demonstrated by examining the effect of
different levels of noise in the input data. The results are
demonstrated also for different measurement scenarios with
amplitude and phase data from near-field or from far-field.
The effect of using simple sensors with only amplitude data is
also studied. Within the range of rotating speeds considered,
the reconstruction can obtain accurate values even for large
noise levels if the speed is large enough. In case of small
rotating speeds, the inversion can give correct values if the
noise levels can be reduced to lower levels. Finally, to show
the generality of the approach, results are provided for a test
case involving rotating torus. For this case, the forward solver
employed uses the finite element method in the absence of an
analytic solver.

The article is organized as follows. In Section \ref{sec:inversion_procedure}
the problem is defined and the two-step algorithm employed for the
inversion procedure is explained in detail. 
In Section \ref{sec:numerical_results} 
the numerical results for the inversion process are provided and
the efficiency and reliability of the two-step procedure are
studied. Finally, the conclusions of the article are discussed
in Section \ref{sec:conclusion}.

\section{Inversion procedure for rotating media} \label{sec:inversion_procedure}

%Problems involving moving media

As mentioned in the previous section, the reconstruction of velocity profiles 
of moving media using electromagnetic inverse scattering techniques is an 
important subject.
Such techniques have been exploited for axially moving cylinders in 
\cite{noiricostruzioneepsrebeta}, \cite{noinoiselimitation}.
Here we study the reconstruction of the rotating speeds of axisymmetric media,
which has numerous practical applications, for example in the tachometry of 
celestial bodies \cite{vanbladel}, \cite{dezutterrotatingsphere}, 
\cite{kalarickel2020well}.

%Explain inverse scattering

The unknown parameters of the inverse scattering problem may be represented by 
the algebraic vector 
${\bf u} = ({\bf u}_g, {\bf u}_d, {\bf u}_v) \in \mathbb{R}^{I+J+K}$, 
where ${\bf u}_g \in \mathbb{R}^I$ , ${\bf u}_d \in \mathbb{R}^J$ and 
${\bf u}_v \in \mathbb{R}^K$ are respectively the components having the 
geometric, dielectric and velocity 
parameters \cite{noiricostruzioneepsrebeta}.

A set of $S$ plane wave sources illuminate the scatterer. 
The positions of the sources are denoted by an integer parameter 
$s = 1, ..., S$. The electric and magnetic fields are measured using $M$ 
sensors whose positions are indicated by the parameter $m = 1, ..., M$.
The measured electric and magnetic fields are denoted respectively by
${\bf E}^{meas}(s, m, {\bf u})$ and ${\bf H}^{meas}(s, m, {\bf u})$.

For a trial solution 
${\bf u}^{trial} = ({\bf u}^{trial}_g, {\bf u}^{trial}_d, {\bf u}^{trial}_v) \in \mathbb{R}^{I+J+K} $, 
we  may assume to have a forward scattering procedure (fsp), either semi analytic 
\cite{dezutterrotatingsphere} or numerical \cite{kalarickel2020well}, that enables
us to calculate the fields ${\bf E}^{fsp}(s, m, {\bf u}^{trial})$ and 
${\bf H}^{fsp}(s, m, {\bf u}^{trial})$.

%Define cost function
The above quantities can be used to define the following cost function, 
which is to be minimized using an optimization algorithm to get the best 
estimate of the unknown parameters:
\begin{align} \label{eq:cost_fuction_general}
	&CF({\bf u}, {\bf u}^{trial}) =  \notag \\
             &\frac{\sum_{s = 1}^S \sum_{m=1}^M ||{\bf E}^{fsp}(s, m, {\bf u}^{trial}) -  {\bf E}^{meas}(s, m, {\bf u})||^2}
		     {\sum_{s = 1}^S \sum_{m=1}^M ||{\bf E}^{meas}(s, m, {\bf u})||^2} \notag \\
		     &+ \frac{\sum_{s = 1}^S \sum_{m=1}^M ||{\bf H}^{fsp}(s, m, {\bf u}^{trial}) -  {\bf H}^{meas}(s, m, {\bf u})||^2}
		     {\sum_{s = 1}^S \sum_{m=1}^M ||{\bf H}^{meas}(s, m, {\bf u})||^2}.
\end{align}
Here the norms are just Euclidean norms for three-dimensional complex vectors.

%Explain the two step procedure
The above steps provide the general way to obtain the parameters using the 
inverse scattering algorithm.
However, if the velocity of the rotating objects is small enough, then we may 
adopt a two-step procedure for the inversion.
Since the contribution to the fields due to the rotation will be small, 
the media may be assumed to be at rest for the reconstruction of geometric and 
dielectric data. Thus, the cost function for the first step will be as follows:

\begin{align} \label{eq:cost_fuction_step1}
	&CF_1({\bf u}, {\bf u}_1^{trial}) =  \notag \\
            &\frac{\sum_{s = 1}^S \sum_{m=1}^M ||{\bf E}^{fsp, rest}(s, m, {\bf u}_1^{trial}) -  {\bf E}^{meas}(s, m, {\bf u})||^2}
		     {\sum_{s = 1}^S \sum_{m=1}^M ||{\bf E}^{meas}(s, m, {\bf u})||^2} \notag \\
		    &+ \frac{\sum_{s = 1}^S \sum_{m=1}^M ||{\bf H}^{fsp, rest}(s, m, {\bf u}_1^{trial}) -  {\bf H}^{meas}(s, m, {\bf u})||^2}
		     {\sum_{s = 1}^S \sum_{m=1}^M ||{\bf H}^{meas}(s, m, {\bf u})||^2}.
\end{align}

Here the vector ${\bf u}^{trial}_1 = ({\bf u}^{trial}_g, {\bf u}^{trial}_d)$ 
contains the geometric and dielectric unknowns and the fields are calculated 
with analytic or numerical solvers for the rest case.

In the second step, we estimate the velocity by optimizing the following 
cost function that uses forward scattering procedure for the rotating media 
along with the approximate geometric and dielectric data ${\bf u}_1^{as}$ 
obtained from the previous step:

\begin{align} \label{eq:cost_fuction_step2}
	&CF_2({\bf u}_, {\bf u}_1^{as}, {\bf u}_v^{trial}) =  \notag \\
             &\frac{\sum_{s = 1}^S \sum_{m=1}^M ||{\bf E}^{fsp}(s, m, {\bf u}_1^{as}, {\bf u}_v^{trial}) -  {\bf E}^{meas}(s, m, {\bf u}_v)||^2}
		     {\sum_{s = 1}^S \sum_{m=1}^M ||{\bf E}^{meas}(s, m, {\bf u}_v)||^2} \notag \\
		     &+ \frac{\sum_{s = 1}^S \sum_{m=1}^M ||{\bf H}^{fsp}(s, m, {\bf u}_1^{as}, {\bf u}_v^{trial}) -  {\bf H}^{meas}(s, m, {\bf u}_v)||^2}
		     {\sum_{s = 1}^S \sum_{m=1}^M ||{\bf H}^{meas}(s, m, {\bf u}_v)||^2}.
\end{align}

Both these steps are much simpler than using the cost function in 
\eqref{eq:cost_fuction_general}. 
For the first step, the forward scattering procedure is much simpler due to the
lack of motion and the optimization is also simpler because the unknowns 
related to the velocity are not present. 
The second step is simpler as well, because it requires
the optimization for only the velocity parameters. 
Since the complexity of the optimization increases much faster than linear, 
the two steps combined can be more efficient than doing
the direct optimization of the full cost function.

%Explain optimizer, parameters used and stopping criteria
For the optimization, here we adopt the differential evolution (DE) algorithm 
\cite{storn1997differential}.
DE is a metaheuristic algorithm that starts with a random initial population 
of size $N_p$ in the search space and
keeps on improving the solution by introducing stochastic variations to 
the candidate solutions until a termination criterion is satisfied.
Here the condition used for termination is that either the maximum number 
of iterations, $N_{lim}$, is reached or the cost function does not improve
more than by a factor of $f_{conv}$ in $N_{conv}$ consecutive iterations.

\section{Numerical results} \label{sec:numerical_results}

In this section, we provide some numerical results for the
reconstruction of the unknown parameters of rotating axisymmetric 
objects from the scattered electromagnetic fields.
The simulations are performed on a Intel i7-8565U, 1.8GHz,
4 core machine with 16 GB RAM. It can be noted that for
the proposed two-step algorithm, the first step involves the
traditional problem of recreating the geometric and dielectric
unknowns for stationary objects and is already widely studied
in the literature \cite{pastorino2010microwave}. 
On the other hand, the second step of the
algorithm for the recreation of the rotating speeds is new
and therefore is analyzed here more carefully. In Subsection 
\ref{subsec:comparison_two_step_with_general}
the proposed two-step procedure is compared to the
general algorithm by solving a test case which has semi
analytic solution. For this, a homogeneous rotating sphere is
considered. The measurement of near-field on a full circle
around the scatterer is used for this and the effect of different
noise levels are also studied. The same test case is used
in Subsections 
\ref{subsec:prefromance_of_two_step} and 
\ref{subsec:reconstruction_from_farfield} 
to study the accuracy of the
two-step procedure with simpler sensors and with far-field
data, respectively. In particular, Subsection 
\ref{subsec:prefromance_of_two_step}
discusses the effect of having scattered field data with the measurement
restricted to be on an arc in the backscattered direction and
when only the amplitude data is available. Subsection 
\ref{subsec:reconstruction_from_farfield}
examines results for the far-field and the effect of uncertainty
in knowing the axis of rotation of the scatterer. 
In Subsection \ref{subsec:effect_of_dielectric_media}, the effect of
changing the dielectric parameter on the accuracy of the algorithm is described.
Finally, in Subsection \ref{subsec:reconstruction_rotating_torus}
a different test case is considered for which no analytic
solution is available. A homogeneous rotating torus is studied
for this and a numerical forward solver is exploited in the
solution procedure.

\subsection{Comparison of the two-step procedure with the general algorithm} \label{subsec:comparison_two_step_with_general}
We consider a simple configuration of a homogeneous sphere
rotating at a uniform speed. A first order semi analytic
solution is available in the literature for such a problem \cite{dezutterrotatingsphere}.
We analyze the performance of the general algorithm as well
as that of the two-step procedure for the reconstruction of the
unknown parameters.

A sphere of radius $R_s = 1$ m is considered which is rotating about 
the z-axis at angular velocity $\omega_s$ rad/s and is
illuminated by a single plane wave ($S=1$) incident along the x-axis with 
a frequency of $f_s = 50$ MHz.
The incident electric field is polarized along the z-axis.
The geometry of the test case is shown in Fig. \ref{fig:geometry_sphere}. 
The speed of rotation can be expressed in terms of the normalized quantity 
$\beta = \omega_s R_s/c_0$, where $c_0$ is the speed of light in vacuum.
The results are for $\varepsilon_r = 8$ and 
$\beta \in \{ 8\times 10^{-3}, 8\times 10^{-4}, 8\times 10^{-5} \}$ and are 
calculated at $M = 200$ points uniformly distributed on a circle on xz plane 
of radius $R_m = 1.5$ m from the center of the scatterer.
The electric and magnetic fields thus obtained are corrupted with a Gaussian 
noise of specified signal to noise ratio (SNR).
We consider SNR dB levels in the set $\{20, 30, 40, 50, 60\}$ added to each 
component of the electric and magnetic fields.

\begin{figure}[h]
\centering
\includegraphics[width=0.5\textwidth]{geometry_rotating_sphere.png}
\caption{The geometry of the test case involving homogeneous spherical
scatterer of radius $R_s$ rotating about the z-axis with angular velocity $\omega_s$. 
It is illuminated by a plane electromagnetic wave of frequency $f_s$ travelling along
the x axis, with electric field polarized along the z axis. The measurement
points are located around the sphere on a circle of radius $R_m$.}
\label{fig:geometry_sphere}
\end{figure}

The inversion is carried out with the DE algorithm with a population 
size of $N_p = 10$.  The parameters related to the termination of the 
algorithm  are set as $N_{lim} = 100$, $f_{conv} = 0.01$ and $N_{conv} = 10$.
The range over which the solutions are searched for is 
$\varepsilon_r \in (1, 20)$ and $\beta \in (-1, 1)$.

\begin{table}
\resizebox{\linewidth}{!}{
\begin{tabular}{|l|l|l|l|l|l|l|l|}
	\hline
	Rotating	     & Algorithm  & \multicolumn{2}{l|}{Mean relative}        & Mean           & Mean number    & Mean             & Mean time     \\
	speed		     &            & \multicolumn{2}{l|}{errors}               & number of      & of function    & minimum          & of simulation \\
	\cline{3-4}
			     &            & $\varepsilon_r$       & $\beta$           & iterations     & evaluations   & cost             & (seconds)     \\
	\hline
			     & General    & $1.39\times 10^{-4}$       & $4.67\times 10^{-3}$   & $40.75$        & $427.50$       & $3.14\times 10^{-4}$  & $7398.0$\\
	$\beta = 8\times 10^{-3}$ & algorithm  &                       &                   &                &                &                  &\\
	\cline{2-8}

			     & Two-step   & $1.12\times 10^{-4}$       & $2.66\times 10^{-3}$   & $16.00$        & $180.00$       & $9.62\times 10^{-3}$  & $1511.1$\\
			     & procedure  &                       &                   & $19.50$        & $215.00$       & $3.04\times 10^{-4}$  & $3037.8$\\
       \hline
			     & General    & $1.79\times 10^{-4}$       & $1.34\times 10^{-1}$   & $33.50$        & $355.00$       & $3.16\times 10^{-4}$  & $6462.8$\\
	$\beta = 8\times 10^{-4}$ & algorithm  &                       &                   &                &                &                  &\\
	\cline{2-8}

			     & Two-step   & $3.74\times 10^{-5}$       & $2.54\times 10^{-2}$   & $19.25$        & $212.50$       & $3.98\times 10^{-4}$  & $1332.0$\\
			     & procedure  &                       &                   & $19.50$        & $215.00$       & $3.04\times 10^{-4}$  & $3595.3$\\
       \hline
			     & General    & $3.05\times 10^{-2}$       & $1.49\times 10^{2}$    & $20.25$        & $222.50$       & $1.00\times 10^{-1}$  & $3419.5$\\
	$\beta = 8\times 10^{-5}$ & algorithm  &                       &                   &                &                &                  &\\
	\cline{2-8}

			     & Two-step   & $1.75\times 10^{-4}$       & $3.43\times 10^{-1}$   & $17.50$        & $185.00$       & $3.16\times 10^{-4}$  & $1607.0$\\
			     & procedure  &                       &                   & $19.25$        & $212.50$       & $3.11\times 10^{-4}$  & $2993.3$\\
       \hline
\end{tabular}}
	\caption{Comparison of the results obtained by the general algorithm 
    and those of the two-step procedure for the reconstruction of 
    the unknown parameters, $\varepsilon_r$ and $\beta$, 
    of rotating sphere. The data is for a SNR level
    of 40 dB in each component of the measured data. 
    The sphere is of radius $R_s = 1$ m, while the illuminating field 
    has a frequency of $f_s = 50$ MHz and a direction of 
    propagation which is perpendicular to the axis of rotation of the 
    sphere.  The actual value of $\varepsilon_r$ is 8. Three values of 
    the speed, $\beta$, are considered and the results are averaged 
    over four test runs.}
\label{table:comparison_of_algorithms}
\end{table}

Table \ref{table:comparison_of_algorithms} summarizes the results for the case 
SNR = 40 dB for the three values of $\beta$ indicated above.
The general algorithm tries to find both the unknowns together using the 
DE algorithm with the cost function in \eqref{eq:cost_fuction_general} and with 
${\bf u} = (\varepsilon_r, \beta)$.
As described in the previous section, the two-step procedure first tries to find 
the solution for ${\bf u}_1 = \varepsilon_r$ by minimizing the cost function in 
\eqref{eq:cost_fuction_step1} by ignoring any rotation of the media.
The approximate solution thus found in the first step is then used in the second 
step to solve for the value of ${\bf u}_2 = \beta$ by minimizing the cost 
function in \eqref{eq:cost_fuction_step2}.

Considering the stochastic nature of the optimization algorithm, all the 
results are averaged over four test runs for each configuration.
As can be seen from the table, for SNR = 40 dB, both the algorithms give good 
results when the value of $\beta = 8\times 10^{-3}$.
The relative errors in $\varepsilon_r$ are 0.0139 percent and 0.0112 percent 
for the general algorithm and the two-step procedure respectively.
The corresponding relative errors in $\beta$ are 0.467 percent and 0.266 percent.
However, the two-step procedure is able to find the solution in 1 hour 
and 15 minutes whereas the general algorithm took 2 hours and 3 minutes to 
find the solution.
The performance of the two-step procedure improves relative to the general 
procedure when the rotating speeds are lower.
For $\beta = 8\times 10^{-4}$, the two-step procedure obtains the solutions 
for $\varepsilon_r$ and $\beta$ with relative errors of 0.00374 percent 
and 2.54 percent respectively.
The corresponding relative errors in the solution obtained from the general 
algorithm are 0.0179 percent and 13.4 percent.
Finally, when $\beta = 8\times 10^{-5}$, the general algorithm give completely 
unreliable result for $\beta$ while the two-step procedure gives a value with 
34.3 percent relative error.

In conclusion, the two-step procedure provides more accurate solutions than 
the general algorithm as the rotating speed gets lower.
The time of simulation is much lower for the two-step procedure than the 
general algorithm. The solutions are reliable for lower values of speed 
as long as the noise level is not too high.
It can be noted that although the two-step
procedure is proposed assuming that the speeds are not too
high, the numerical results show that it is applicable even
for the highest of the velocity values that can be of practical
relevance.

To get an indication of how larger values of rotating speed can affect the 
accuracy of the two-step algorithm, the error values from both the algorithms 
are provided in Fig. \ref{fig:error_beta_vs_beta_comparison_of_algorithms} 
for $\beta$ values in the set 
$\{4\times 10^{-3}, 8\times 10^{-3}, 1.6\times 10^{-2}, 3.2\times 10^{-2},
6.4\times 10^{-2}, 1.28\times 10^{-1}\}$.
The results indicate that, for the homogeneous medium considered 
($\varepsilon_r=8$), the two-step algorithm performs better than the 
general algorithm for values of $\beta$ that are less than around 
$2.75 \times 10^{-2}$. Although the exact values can change with the dielectric
medium, the results confirm that the two-step algorithm provides accurate 
results for all practical speeds that can be of interest.

\begin{figure}[h]
\centering
\includegraphics[width=0.5\textwidth]{figure_error_vs_beta_sphere_comparison_between_algorithms.eps}
\caption{The mean relative error in the reconstructed value of $\beta$ vs the value of $\beta$  for the rotating sphere. 
         The result obtained from the two-step algorithm is compared with that obtained using the general algorithm.
         The measured fields are corrupted with SNR level of 40 dB.}
\label{fig:error_beta_vs_beta_comparison_of_algorithms}
\end{figure}

To better understand how the results of the two-step procedure 
are affected by the noise in the scattered field data 
we give the effect of SNR values on the error in the reconstructed speed
obtained using the two-step procedure.
In Fig. \ref{fig:error_beta_vs_SNR} the mean relative error in the 
reconstructed value of $\beta$ is plotted against the SNR levels in dB.
The plots for three rotating speeds, 
$\beta \in \{  8\times 10^{-3}, 8\times 10^{-4}, 8\times 10^{-5}\}$, are shown.
For $\beta =  8\times 10^{-3}$, the relative error is  just 0.032 percent for 
SNR = 60 dB and it increases to 8.2 percent when SNR = 20 dB.
When $\beta = 8\times 10^{-4}$, the average relative error values 
are 0.35 percent for SNR = 60 dB and 27 percent  for SNR = 20 dB.
Finally, for $\beta = 8\times 10^{-5}$, the average relative errors are 
4.47 percent for SNR = 60 dB and 493.6 percent for SNR = 20dB.
It can be noted that for small values of $\beta$ the effects of motion
on the electromagnetic field are small and the noise can easily
affect the field more than motion. In these cases accurate
measurements of the field are necessary. It is interesting to
point out, however, that if this is done good results can be
achieved.

\begin{figure}[h]
\centering
\includegraphics[width=0.5\textwidth]{figure_beta_error_vs_SNR_sphere_rotating.eps}
\caption{The mean relative error in the reconstructed value of $\beta$ vs the SNR levels in dB for the rotating sphere. 
         The results are shown for three different values of $\beta$.}
\label{fig:error_beta_vs_SNR}
\end{figure}

\subsection{Performance of two-step procedure with simpler sensors} \label{subsec:prefromance_of_two_step}

In the previous, subsection we examined the performance of the reconstruction 
algorithm when the measurement of the complex fields was available on the full 
circle outside the scatterer.
Now we examine the performance of the two-step algorithm under limitations 
introduced by the sensors.
First we study the effect of restricting the measurement points to only 
a small portion along the backscattering direction.
After that, we also investigate the effect of having only the amplitude data 
for the electromagnetic fields which can allow us to use much simpler sensors.

\begin{figure}[h]
\centering
\includegraphics[width=0.5\textwidth]{figure_beta_error_vs_SNR_sphere_rotating_backscatt.eps}
\caption{The mean relative error in the reconstructed value of $\beta$ vs the 
         SNR levels in dB for the rotating sphere using data in the 
         backscattered direction. The results are shown for three different 
         values of $\beta$.}
\label{fig:error_beta_vs_SNR_backscatt}
\end{figure}

\begin{figure}[h]
\centering
\includegraphics[width=0.5\textwidth]{figure_beta_error_vs_SNR_sphere_rotating_backscatt_amplitude_only.eps}
\caption{The mean relative error in the reconstructed absolute value of 
         $\beta$ vs the SNR levels in dB for the rotating sphere using 
         amplitude only data in the backscattered direction. The results are 
         shown for three different values of $\beta$.}
\label{fig:error_beta_vs_SNR_backscatt_backscatt_amplitude_only}
\end{figure}

The results obtained by considering only the measurement points along the 
backscattered direction are provided in 
Fig. \ref{fig:error_beta_vs_SNR_backscatt}.
For this we use only the portion of measurement data that belongs to an arc 
subtending an angle of 90 degrees along the backscattering direction.
So we considered 50 measurement points on an arc on xz plane of radius 1.5 m 
from the center of the scatterer. It is observed that the reconstruction 
algorithm still works well with such a restriction.
For example for $\beta = 8\times 10^{-4}$ and with SNR = 60 dB, the relative 
error in the reconstructed value of $\beta$  is  0.55 percent as compared 
to 0.35 percent obtained previously using the data on the full circle. 
For $\beta = 8\times 10^{-4}$ and SNR = 40 dB, the corresponding values are 
3.4 percent and 2.5 percent.
With  $\beta = 8\times 10^{-4}$ and SNR = 20 dB the errors become 108.1 
percent using the data along backscattered direction  as opposed to 27.3 
percent with the data on the full circle.

Next, we examine the effect of using sensors that can measure only the 
amplitude of the electromagnetic fields.
For this case, we have to modify the cost functions involved so that only 
differences in the magnitudes of each component of the fields are considered.
The measurement points used for the reconstruction are the same as those in 
the previous step involving only backscattered fields.
In this case, it is not possible to distinguish between the clockwise and 
anticlockwise rotations and therefore we have to consider only the absolute 
value of the reconstructed $\beta$.
The results are provided in 
Fig. \ref{fig:error_beta_vs_SNR_backscatt_backscatt_amplitude_only}.
As expected, the accuracy is not as good as that obtained with both 
amplitude and phase information.
For example, with $\beta = 8\times 10^{-4}$ and with SNR = 40 dB, the error 
in the reconstructed speed is 64.1 percent as opposed to 3.4 percent 
obtained using both amplitude and phase measurements.
However, with less noisy data the accuracy is acceptable as in the case with
$\beta = 8\times 10^{-4}$ and with SNR = 60 dB, where we get an  error in the 
reconstructed speed of 1.28 percent.
Similar comparisons can be made for the case of $\beta = 8\times 10^{-3}$ and 
$\beta = 8\times 10^{-5}$ shown in the figures and the results are good when 
the speed is not very small and the SNR is good enough.

Therefore, in conclusion, we still get good solutions for the rotating speeds 
when the measurement is restricted to the backscattering direction.
The results are degraded when only amplitude data are available but 
the errors are still small provided that the noise does not overwhelm
the effects of motion.

\subsection{Reconstruction of the speed of rotation from far-field data} \label{subsec:reconstruction_from_farfield}
So far, we concentrated on the performance of the reconstruction algorithm 
using the total field data in the near-field region.
Next, let us analyze the performance of the algorithm with the scattered 
electromagnetic field in the far-field region.
For this, we consider the measurement arc on xz plane in the backscattered 
direction of radius $1.5 \times 10^{3}$ m from the center of the scatterer.
The results are plotted in Fig. \ref{fig:error_beta_vs_SNR_farfield}.

\begin{figure}[h]
\centering
\includegraphics[width=0.5\textwidth]{figure_beta_error_vs_SNR_sphere_rotating_farfield.eps}
\caption{The mean relative error in the reconstructed value of $\beta$ vs the 
         SNR levels in dB for the rotating sphere using far-field data in 
         the backscattered direction. The results are shown for three 
         different values of $\beta$.}
\label{fig:error_beta_vs_SNR_farfield}
\end{figure}

The algorithm performs well for $\beta = 8\times 10^{-3}$ and the error for 
SNR values greater than or equal to 30 dB is less than 2.2 percent and is 
13.1 percent for SNR of 20 dB.
However, for $\beta = 8\times 10^{-4}$, the error in the reconstructed value 
of $\beta$ is 3.7 percent for SNR of 60 dB and is around 9 percent for 40 dB 
and rises to 23 percent for 30 dB SNR.
Finally, for  $\beta = 8\times 10^{-5}$, the reconstruction algorithm performs 
very poorly for the considered values of SNR. 
The error for 60 dB SNR in the data is 4 percent while for 50 dB the error is 
19.3 percent and rises to unacceptable levels as the noise increases.

Let us also examine the effect of error in the knowledge of the axis of 
rotation of the scatterer.
For this, we may generate the measurement data using the forward model on an 
arc whose plane is rotated by an angle $\alpha$ from the xz plane while still 
using the old set of points for the reconstruction algorithm.
The results with such an approach are shown in 
Fig. \ref{fig:error_beta_vs_SNR_farfield_error_10_degree} 
for $\alpha = 10^{\circ}$.
For $\beta = 8\times 10^{3}$, the algorithm is able to find the value of the
rotating speed with an error of less than 1.6 percent for SNR values greater 
than or equal to 30 dB.
The error for $\beta = 8\times 10^{-4}$ is 1.3 percent for SNR of 60 dB, 
15.1 percent for SNR of 40 dB  and 22.8 percent for SNR of 30 dB.
Finally, for $\beta = 8 \times 10^{-5}$, the results for the considered 
values of SNR are unreliable with a 36 percent error even for SNR of 60 dB, 
which rises to a 102 percent error for SNR of 40 dB.

\begin{figure}[h]
\centering
\includegraphics[width=0.5\textwidth]{figure_beta_error_vs_SNR_sphere_rotating_farfield_error_10_degree.eps}
	\caption{The mean relative error in the reconstructed value of $\beta$ vs 
    the SNR levels in dB for the rotating sphere using far-field data
	in the backscattered direction and $\alpha = 10^{\circ}$.
	The results are shown for three different values of $\beta$.}
\label{fig:error_beta_vs_SNR_farfield_error_10_degree}
\end{figure}

\subsection{The effect of different dielectric media} \label{subsec:effect_of_dielectric_media}

Let us also examine the effect of changing the dielectric parameter, 
$\varepsilon_r$, on the accuracy of the reconstructed speed. 
The result for a rotating speed of $\beta = 8 \times 10^{-4}$ is presented in 
Fig. \ref{fig:error_beta_vs_epsilonr_beta_8e-4}.
The measurements are considered at $M = 200$ points on a circle of 
radius $R_m = 1.5$ m. 
The permittivity values considered are $\varepsilon_r \in \{4, 8, 12, 16, 20\}$, 
and the results are shown for three noise levels, SNR $\in \{30, 40, 50\}$ (dB).
The reconstruction becomes more difficult if the scattering is weak and 
therefore we can in general expect a larger error for smaller values of 
$\varepsilon_r$.  This trend can be confirmed from the obtained results.
For instance, with SNR = 40 dB, the relative error is 25.7 percent for 
$\varepsilon_r =4$ while it reduces to 0.37 percent for $\varepsilon_r = 20$.
Similarly, the error goes from 140 percent to 1.5 percent 
for SNR = 30 dB and from 2.99 percent to 0.069 percent for SNR = 50 dB,
as $\varepsilon_r$ increases from 4 to 20.

\begin{figure}[h]
\centering
\includegraphics[width=0.5\textwidth]{figure_beta_error_vs_epsilonr_sphere_rotating_beta_8e-4.eps}
	\caption{The mean relative error in the reconstructed value of $\beta$ 
             vs $\varepsilon_r$ for the rotating sphere with 
             $\beta = 8\times 10^{-4}$ and the measurements 
             are considered at $M=200$ points on a circle of $R_m = 1.5$ m. 
             The results are shown for three different values of SNR.}
\label{fig:error_beta_vs_epsilonr_beta_8e-4}
\end{figure}

\subsection{Reconstruction of the speed of rotating torus} \label{subsec:reconstruction_rotating_torus}
All the previous results were obtained for rotating sphere which admits 
a semi analytic solution for the forward problem.
Now we apply the algorithm to rotating homogeneous torus for which analytic 
solutions are not available; 
a numerical forward solver has to be exploited instead.
The direct solution for the fields is obtained using a finite element method 
\cite{jin}, \cite{kalarickel2020well}.
The solver is implemented using a model in the commercial simulator 
COMSOL Multiphysics (COMSOL, Burlington, MA).
The axis of symmetry of the torus is taken as the z-axis about which the torus 
rotates with an angular velocity of $\omega_s$.
The major radius, $R_{tor}$, and the minor radius, $r_{tor}$, of the torus 
\cite{kalarickel2020well} are taken as respectively 0.75 m and 0.25 m. 
The geometry of the test case is shown in Fig. \ref{fig:geometry_torus}.
We may express the speed of rotation using the normalized quantity 
$\beta = \frac{\omega_s(R_{tor} + r_{tor})}{c_0}$. The scattering medium at 
rest is characterized by $\varepsilon_r = 8$ and $\mu_r = 1$.
It is illuminated by a plane wave incident along the x axis, polarized along 
z-axis and with a frequency $f_s = 150$ MHz.

\begin{figure}[h]
\centering
\includegraphics[width=0.5\textwidth]{geometry_rotating_torus.png}
\caption{The geometry of the test case involving homogeneous toroidal
         scatterer of major radius $R_{tor}$ and minor radius $r_{tor}$ rotating about the z axis
         with angular velocity $\omega_s$. It is illuminated by a plane electromagnetic wave of
         frequency $f_s$ travelling along the x axis, with electric field polarized along the z
         axis.}
\label{fig:geometry_torus}
\end{figure}

The numerical domain is set to $R_d = 4$ m and the measurement data are taken 
at $M = 200$ uniformly placed points on a circle in xz plane of radius 
$R_m = 1.5$ m. The variational formulation is implemented in the mathematics 
module of the COMSOL Multiphysics using a second order edge element formulation.
A tetrahedral meshing is done with 241181 elements, 41605 nodes and 
9438 boundary elements.  The algebraic solver used is GMRES with a tolerance 
of $10^{-4}$ and a with geometric multigrid preconditioner.
The reconstruction algorithm searches for the optimal value of $\beta$ in the 
range $(0, 10^{-2})$ and the parameters of the DE optimizer are the same as 
before ($N_p = 10$, $N_{lim} = 100$, $f_{conv} = 0.01$, $N_{conv} = 10$).
The measured data is corrupted with noise as before with SNR levels varying 
from 20 dB to 60 dB.  Compared to the analytical solutions, the numerical 
solvers produce additional noise, due to discretization errors and 
the tolerance in algebraic solver, which could affect the performance of the 
inverse algorithm.

The results are shown in Fig. \ref{fig:error_beta_vs_SNR_torus} for values 
of $\beta \in \{8 \times 10^{-3}, 8 \times 10^{-4}, 8 \times 10^{-5}\}$.
For $\beta = 8\times 10^{-3}$, the relative error in the solution is less than 
1.6 percent when SNR in the measured data is greater than or equal to 30 dB.
In the case of $\beta = 8\times 10^{-4}$, the relative errors are 2.14 percent 
for 50 dB SNR, 13.9 percent for 40 dB SNR and 21.6 percent for 30 dB SNR.
When $\beta = 8\times 10^{-5}$, the relative errors are very high.
It is 16.7 percent for 60 dB SNR, 29.9 percent for 50 dB SNR and more than 
100 percent when SNR is greater than or equal to 40 dB.
As mentioned before, the additional noise in the numerical solver is a reason 
for the worse performance of the reconstruction algorithm compared to
the case with semi analytic forward solver.
In particular, for small values of the
rotating speed, we need not only to perform very accurate
measurements of the fields but also to reduce numerical errors
in order to get reliable results.

\begin{figure}[h]
\centering
\includegraphics[width=0.5\textwidth]{figure_beta_error_vs_SNR_torus_rotating.eps}
	\caption{The mean relative error in the reconstructed value of $\beta$ 
    vs the SNR levels in dB for the rotating torus
	with measurement data on a circle of radius $R_m = 1.5$ m on the xz plane.
	The results are shown for three different values of $\beta$.}
\label{fig:error_beta_vs_SNR_torus}
\end{figure}


\section{Conclusion} \label{sec:conclusion}
In this paper, the electromagnetic inverse problem involving
rotating axisymmetric objects was studied for the first time.
We investigated the reconstruction of the rotating speeds of
axisymmetric objects from the scattered field data. A two-step 
algorithm was proposed for the inversion when the
speeds are known to be not too large. In the first step, a forward
solver is employed assuming zero rotating speed and the
geometric and dielectric data are reconstructed by minimizing
the cost function. In the second step, the values from the first
step are used to determine the rotating speed. By defining
a test case involving rotating homogeneous sphere, the two-
step procedure is compared with the straightforward inversion 
procedure which solves for all the unknowns together.
It was established that the two-step procedure is much faster
and also more accurate if the rotating speeds assume values
of practical interest. The performance of the algorithm was
tested with different types of noisy data: near-field data,
far-field data and amplitude only data were used as 
inputs to the two-step procedure to obtain the rotating speeds.
The results show that for relatively small speed values the
two-step procedure can reliably recover the velocity of rotation 
provided that the noise does not overwhelm
the effects of the rotation on the field. In addition to rotating
sphere, the algorithm was also applied to a rotating torus to
demonstrate the generality of the results. This article studied
only the test cases with homogeneous velocity profiles to give
the first results on the efficiency and accuracy of the two-step
procedure. However, the proposed algorithm is general and
can be applied to more complex velocity profiles.

\bibliographystyle{IEEEtran}
\bibliography{myreferences}
\EOD
\end{document}

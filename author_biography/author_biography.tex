\documentclass{ieeeaccess}
\usepackage{cite}
\usepackage{amsmath,amssymb,amsfonts}
\usepackage{algorithmic}
\usepackage{graphicx}
\usepackage{textcomp}
\def\BibTeX{{\rm B\kern-.05em{\sc i\kern-.025em b}\kern-.08em
    T\kern-.1667em\lower.7ex\hbox{E}\kern-.125emX}}
\begin{document}
\history{Date of publication xxxx 00, 0000, date of current version xxxx 00, 0000.}
\doi{10.1109/ACCESS.2017.DOI}

\title{Preparation of Papers for IEEE ACCESS}
\author{\uppercase{First A. Author}\authorrefmark{1}, \IEEEmembership{Fellow, IEEE},
\uppercase{Second B. Author\authorrefmark{2}, and Third C. Author,
Jr}.\authorrefmark{3},
\IEEEmembership{Member, IEEE}}
\address[1]{National Institute of Standards and 
Technology, Boulder, CO 80305 USA (e-mail: author@boulder.nist.gov)}
\address[2]{Department of Physics, Colorado State University, Fort Collins, 
CO 80523 USA (e-mail: author@lamar.colostate.edu)}
\address[3]{Electrical Engineering Department, University of Colorado, Boulder, CO 
80309 USA}
\tfootnote{This paragraph of the first footnote will contain support 
information, including sponsor and financial support acknowledgment. For 
example, ``This work was supported in part by the U.S. Department of 
Commerce under Grant BS123456.''}

\markboth
{Author \headeretal: Preparation of Papers for IEEE TRANSACTIONS and JOURNALS}
{Author \headeretal: Preparation of Papers for IEEE TRANSACTIONS and JOURNALS}

\corresp{Corresponding author: First A. Author (e-mail: author@ boulder.nist.gov).}

\begin{IEEEbiography}[{\includegraphics[width=1in,height=1.25in,clip,keepaspectratio]{praveen_kr.png}}]{Praveen Kalarickel Ramakrishnan} 
received bachelor’s degree in electrical and Electronics Engineering from 
National Institute of Technology, Calicut, India in 2010, master’s degree in 
electrical engineering from Indian Institute of Science, Bangalore, 
India in 2012 and Ph.D. in Electronics and Telecommunication Engineering 
from University of Genoa, Italy in 2018. He worked as a Research and Development 
Engineer in Tribi Systems from 2012 to 2014 and as a postdoctoral researcher 
at the University of Genoa from 2018 to 2021.
His research interests include direct and inverse scattering, computational
electromagnetics, RF and microwave engineering and bioelectromagnetics.
\end{IEEEbiography}

\begin{IEEEbiography}[{\includegraphics[width=1in,height=1.25in,clip,keepaspectratio]{mario_clemente.png}}]{Mario Rene Clemente Vargas} 
received the B.Sc. degree in Electronic Engineering from the National 
University of Engineering (Peru) in 2002 and M.Sc degree from the University of
Genoa (Italy) in 2014. He is a Ph.D. student at the
Department of Electrical, Electronic, Telecommunications Engineering, 
and Naval Architecture of the University of Genoa. He worked as technical
support in the field of telecommunications. Currently, he is working 
at the Applied Electromagnetics Laboratory of the Department of 
Electrical, Electronic, Telecommunications Engineering, and Naval 
Architecture (DITEN) of the University of Genoa. 
His main research activities focus on the field of high-capacity
transmission lines for microwave circuits and photobiomodulation therapy. 
\end{IEEEbiography}

\begin{IEEEbiography}[{\includegraphics[width=1in,height=1.25in,clip,keepaspectratio]{mirco_raffetto.png}}]{Mirco Raffetto}  
was born in Genoa, Italy, in 1967. He received the “laurea” degree in 
Electronic Engineering (“summa cum laude”) from the University of Genoa 
in 1990, and the Ph. D. degree in “Models, Methods and Tools for Electronic and
Electromagnetic Systems” from the same university in 1997. 
After a four-year period spent in a telecommunications company, in 1994 he 
joined the Applied Electromagnetic Group, University of Genoa, where, 
at present, he is a Full Professor of Electromagnetic Fields at the 
Department of Electrical, Electronic, Telecommunications Engineering, 
and Naval Architecture. Since 2016 he is the coordinator of the bachelor 
of science course in Electronic Engineering and Information Technology. 
His main research interests are in electromagnetic theory, 
electromagnetic scattering and computational electromagnetics. In
2006, he was the recipient of the Emerald Literati Network outstanding paper
award
\end{IEEEbiography}

\EOD

\end{document}
